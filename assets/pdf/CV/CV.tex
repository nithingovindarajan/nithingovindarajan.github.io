%------------------------------------
% Dario Taraborelli
% Typesetting your academic CV in LaTeX
%
% URL: http://nitens.org/taraborelli/cvtex
% DISCLAIMER: This template is provided for free and without any guarantee 
% that it will correctly compile on your system if you have a non-standard  
% configuration.
% Some rights reserved: http://creativecommons.org/licenses/by-sa/3.0/
%------------------------------------

%!TEX TS-program = xelatex
%!TEX encoding = UTF-8 Unicode

\documentclass[10pt, a4paper]{article}


% DOCUMENT LAYOUT
\usepackage{geometry} 
\usepackage{hyperref}
\usepackage{changepage}
\geometry{a4paper, textwidth=5.5in, textheight=8.5in, marginparsep=1pt, marginparwidth=1.1in}
\setlength\parindent{0in}

% FONTS
% ---- CUSTOM AMPERSAND
\newcommand{\amper}{{\fontspec[Scale=.95]{Fontin}\selectfont\itshape\&}}
% ---- MARGIN YEARS
\usepackage{marginnote}
\newcommand{\years}[1]{\marginnote{\scriptsize #1} }
\renewcommand*{\raggedleftmarginnote}{}
\setlength{\marginparsep}{7pt}
\reversemarginpar

% HEADINGS
\usepackage{sectsty} 
\usepackage[normalem]{ulem} 
\sectionfont{\rmfamily\mdseries\upshape\Large}
\subsectionfont{\rmfamily\bfseries\upshape\normalsize} 
\subsubsectionfont{\rmfamily\mdseries\upshape\normalsize} 

% PDF SETUP
% ---- FILL IN HERE THE DOC TITLE AND AUTHOR

% DOCUMENT
\begin{document}
{\LARGE Nithin Govindarajan}\\[1cm]
Phone: \texttt{+31 6 57156899 }\\
email: {nithin.govindarajan@kuleuven.be}\\
Location: Leuven, Belgium \\
website: \url{https://nithingovindarajan.github.io/}


%%\hrule
\section*{Areas of expertise}
Numerical linear algebra, tensor methods, dynamical systems, systems theory, signal processing.
%\hrule
\section*{Education}

\years{Oct. 2014 - Dec. 2018}\textsc{PhD} in Mechanical Engineering, University of California in Santa Barbara
\begin{adjustwidth}{2.5em}{0pt}
	\emph{Dissertation title:} ``Periodic approximations and spectral analysis of the Koopman operator: theory and applications''. \\
	\emph{Advisors:} I. Mezi\'{c}, S. Chandrasekaran. 
\end{adjustwidth}
\years{Sep. 2009 - Oct. 2012}\textsc{MSc} in Aerospace Engineering (with distinction), Technische Universiteit Delft 
\begin{adjustwidth}{2.5em}{0pt}
	\emph{Master thesis:} ``An Optimal Control Approach for Estimating Aircraft Command Margins''. \\
	\emph{Advisors:} Q.P.  Chu, C.C. de Visser. 
\end{adjustwidth}
\years{Jul. 2009}\textsc{BSc} in Aerospace Engineering (with distinction), Technische Universiteit Delft 

\section*{Work experience}
\years{Aug. 2019 - present} Postdoctoral researcher, KU Leuven, Belgium \\
\years{Feb. 2019 - July 2019} Lecturer Mathematics, University of Amsterdam (UvA), The Netherlands \\
\years{Nov. 2012 - May 2013} Junior R\&D engineer, National Aerospace Laboratory, Amsterdam, The Netherlands \\
\years{Sep. 2011 - May 2012} Intern, Mission Critical Technologies Inc. (on site at NASA Ames), Moffet Field, CA 

%\hrule
\section*{Fellowships, honors \& awards}
\noindent

\years{Apr. 2016} CCDC fellowship, Center for Control, Dynamics and Computation, Santa Barbara, CA \\
\years{Apr. 2014} Department Merit Fellowship (UCSB Mech. Eng), Santa Barbara, CA \\
% \years{Mar. 2013} Fulbright scholarship (awarded), Fulbright office, Amsterdam, The Netherlands \\
\years{May 2011} Huygens Scholarship Programme, Nuffic, The Hague, The Netherlands





\section*{Publications}



\subsection*{Journal Publications}

\years{2022}Govindarajan, N., Vervliet, N., \& De Lathauwer, L. (2022). Regression and classification with spline-based separable expansions. Frontiers in big Data, 5, 688496.  \\
\years{2022}Govindarajan, N., Epperly, E. N., \& Lathauwer, L. D. (2022). ($L_r,L_r,1$)-Decompositions, Sparse Component Analysis, and the Blind Separation of Sums of Exponentials. SIAM Journal on Matrix Analysis and Applications, 43(2), 912-938. \\
\years{2021}Epperly, E. N., Govindarajan, N., \& Chandrasekaran, S. (2021). Minimal rank completions for overlapping blocks. Linear Algebra and its Applications, 627, 185-198.  \\
\years{2021}Govindarajan, N., Mohr, R., Chandrasekaran, S., \& Mezic, I. (2021). On the approximation of Koopman spectra of measure-preserving flows. SIAM Journal on Applied Dynamical Systems, 20(1), 232-261. \\
\years{2019}Govindarajan, N., Mohr, R., Chandrasekaran, S., \& Mezic, I. (2019). On the approximation of Koopman spectra for measure preserving transformations. SIAM Journal on Applied Dynamical Systems, 18(3), 1454-1497. \\
\years{2015}Govindarajan, N., De Visser, C. C., Van Kampen, E., Krishnakumar, K., Barlow, J., \& Stepanyan, V. (2015). Optimal control framework for estimating autopilot safety margins. Journal of Guidance, Control, and Dynamics, 38(7), 1197-1207. \\
\years{2014} Govindarajan, N., de Visser, C. C., \& Krishnakumar, K. (2014). A sparse collocation method for solving time-dependent HJB equations using multivariate B-splines. Automatica, 50(9), 2234-2244.  

\subsection*{Conference proceedings}
\years{2023} Widdershoven, R., Govindarajan, N., De Lathauwer, L. (2023, September). Overdetermined systems of polynomial equations: tensor-based solution and application. Proceedings of EUSIPCO 2023, Helsinki, Finland. \\
\years{2018} Chandrasekaran, S., Govindarajan, N., \& Rajagopal, A. (2018, July). Fast Algorithms for Displacement and Low-Rank Structured Matrices. In Proceedings of the 2018 ACM International Symposium on Symbolic and Algebraic Computation (pp. 17-22). \\
\years{2016} Govindarajan, N., Arbabi, H., Van Blargian, L., Matchen, T., \& Tegling, E. (2016, December). An operator-theoretic viewpoint to non-smooth dynamical systems: Koopman analysis of a hybrid pendulum. In 2016 IEEE 55th Conference on Decision and Control (CDC) (pp. 6477-6484). IEEE.  

\subsection*{Preprints \& Tech reports}
\years{2023} Govindarajan, N., Widdershoven, R., Chandrasekaran, S. (2023). A fast algorithm for computing Macaulay nullspaces of bivariate polynomial systems. ESAT Tech Report 23-16.  \\
\years{2019} Chandrasekaran, S., Epperly, E. N., Govindarajan, N. (2019). Graph-induced rank structures and their representations. arXiv preprint arXiv:1911.05858 [math.NA].   

\subsection*{Selected talks}
\noindent
\years{2023} ``A tensor-based approach to solving systems of multivariate polynomials'', CAM23, Selva di Fasano. \\
\years{2023} ``Efficient Computation of Macaulay Matrix Null Spaces Through Exploiting Shift-Invariant Structures'', SIAM AG23, Eindhoven. \\
\years{2021} 
“$(L_r,L_r,1)$-decompositions, Sparse Component Analysis, and the Blind Separation of Sums of Exponentials”, SeLMA meeting, Leuven.
\years{2018} ``Spline-based separable expansions for approximation, regression and classification'',  IPAM Workshop I: Tensor Methods and their Applications in the Physical and Data Sciences, UCLA (online) \\ 
\years{2017} ``A toolbox for computing spectral properties of dynamical systems'', SIAM DS17, Snowbird.  
\years{2016} ``An operator-theoretic viewpoint to non-smooth dynamical systems: Koopman analysis of a hybrid pendulum'', IEEE CDC 16, Las Vegas.


%%\hrule
\section*{Teaching}
\subsection*{Lead instructor \& course organizor}
\years{Semester 1 2019/2020} Numerical mathematics, Amsterdam University College \\

\subsection*{Co-instructor}
\years{March 2022} Fast algorithms for dense structured matrices, KU Leuven \\
\years{Semester 2 2021/2022} Numerieke modellering \& benadering, KU Leuven \\
\years{Semester 2 2019/2020} Numerieke modellering \& benadering,  KU Leuven \\
\subsection*{Teaching assistant}
\years{Semester 1 2021/2022} Numerieke wiskunde, KU Leuven \\
\years{Semester 1 2020/2021} Numerieke wiskunde, KU Leuven \\
\years{Semester 1 2019/2020} Numerieke wiskunde, KU Leuven \\
\years{Spring 2018} Control theory, UCSB \\
\years{Winter 2017} Electrical circuits Lab, UCSB \\
\years{Fall 2017}   Intro to programming in Matlab, UCSB\\
\years{Summer 2017} Dynamics, UCSB\\
\years{Summer 2017} Physics Lab: intro to classical mechanics for non-engineers, UCSB\\
\years{Spring 2017} Dynamics, UCSB\\
\years{Spring 2015} Dynamics, UCSB\\
\years{Winter 2015} Vibrations, UCSB\\
\years{Fall 2014} Statics, UCSB

\section*{Extra-curricular}
\noindent


\years{Sept. 2019 - present} Co-founder and technical brain of software startup \url{www.matisse.ai} in dental technologies.



\section*{Software skills}
Matlab, Python, Julia, Mathematica (basic), C++ (basic), Latex, Git.


\section*{Languages}
English, Dutch, Tamil (basic).


%\vspace{1cm}
\vfill{}
%\hrulefill

\begin{center}
{\scriptsize  Last updated: \today}
\end{center}

\end{document}